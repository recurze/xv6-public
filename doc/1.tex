\documentclass[12pt]{article}

\usepackage{amsmath}
\usepackage{centernot}
\usepackage{lastpage}
\usepackage{fancyhdr}
\usepackage[most]{tcolorbox}
\usepackage{geometry}
\usepackage{graphicx}
\usepackage{subcaption}
\usepackage{listings}
\usepackage{color}


\geometry{
    top=3cm,
    bottom=3cm,
    left=2.5cm,
    right=2.5cm,
}

\pagestyle{myheadings}
\pagestyle{fancy}
\fancyhf{}

\setlength{\headheight}{10pt}

\renewcommand{\headrulewidth}{4pt}
\renewcommand{\footrulewidth}{2pt}

\fancyhead[L]{COL331: Assignment 1}
\fancyhead[C]{}
\fancyhead[R]{xv6}
\fancyfoot[L]{}
\fancyfoot[C]{}
\fancyfoot[R]{\thepage/\pageref{LastPage}}

\pagenumbering{roman}
\author{\textbf{Rahul V}}
\date{}
\title{
    \thispagestyle{empty}
    \vspace{0.2\textheight}
    \textbf{COL331: Lab 1} \\
    \vspace{0.01\textheight}
    \large \emph{Instructor: Smruti R. Sarangi} \\
    \vspace{0.01\textheight}
    \small Due\ on\ February 25, 2019.
    \vspace{0.2\textheight}
}

\definecolor{dkgreen}{rgb}{0,0.6,0}
\definecolor{gray}{rgb}{0.5,0.5,0.5}
\definecolor{mauve}{rgb}{0.58,0,0.82}

\lstset{frame=tb,
  language=Java,
  aboveskip=3mm,
  belowskip=3mm,
  showstringspaces=false,
  columns=flexible,
  basicstyle={\small\ttfamily},
  numbers=none,
  numberstyle=\tiny\color{gray},
  keywordstyle=\color{blue},
  commentstyle=\color{dkgreen},
  stringstyle=\color{mauve},
  breaklines=true,
  breakatwhitespace=true,
  tabsize=3
}

\newcommand{\genTitlePage} {
    \maketitle
    \begin{center}
        \vspace{-0.05\textheight}
        \large 2016CS10370
    \end{center}
}
\newcommand{\BigO}[1]{\ensuremath{\mathcal{O}\left(#1  \right)}}

\begin{document}

\genTitlePage
\clearpage
\pagenumbering{arabic}

\section{Adding $syscalls$ to xv6}
\subsection {$Syscalls$ added}
\begin{itemize}
    \item \texttt{print\_count}: prints the number of times each \texttt{syscall}
        was invoked since state changed to \texttt{TRACE\_ON}.
    \item \texttt{toggle}: toggles the \texttt{trace\_state} and resets the
        counter for each \texttt{syscall}.
    \item \texttt{add}: adds two integer arguments and returns their sum.
    \item \texttt{ps}: prints all current processes, i.e., all the processes
        which are not in \texttt{UNUSED} state.
    \item \texttt{send}: sends \texttt{msg} from process \texttt{sender\_pid} to
        process \texttt{rec\_pid}.
    \item \texttt{recv}: receives \texttt{msg} from mailbox. This is a blocking
        call.
    \item \texttt{send\_multi}: sends \texttt{msg} from process
        \texttt{sender\_pid} to list of processes \texttt{rec\_pid[]}. An interrupt
        is sent to each of the \texttt{rec\_pid[i]} to notify about new \texttt{msg}.
\end{itemize}
\subsection{How to add a $syscall$}
Adding a $syscall$ requires changes made in multiple files, namely:
\begin{itemize}
    \item \texttt{defs.h}: add a forward declaration for functions to be defined
        in \texttt{proc.c}
    \item \texttt{user.h}: declare the function that can be called through shell,
        i.e., name of the $syscall$.
    \item \texttt{usys.S}: Function call is then linked to the respective
        $syscall$ using the macro in this file.
    \item \texttt{syscall.h}: define the position of the syscall vector, syscall
        number.
    \item \texttt{syscall.c}: \texttt{extern} declare the function to be
        defined in \texttt{sysproc.c} and add function to syscall vector.
    \item \texttt{sysproc.c}: add the real implementation of the $syscall$.
    \item \texttt{user\_prog.c}: Handler function for the new $syscall$.
\end{itemize}

\newpage
\subsection{Example}
\begin{lstlisting}
// File: user.h
int add(int, int);

// File: usys.S
SYSCALL(add) // SYSCALL is the macro defined in the same file and mentioned above


// File: syscall.h
#define SYS_add 1 // ascending order is you will

// File: syscall.c
extern int sys_add(void); // arguments are taken from stack.
static int (*syscalls[])(void) = {
[SYS_add]    sys_add,
. . .
. . .
. . .
. . .
};

// File: sysproc.c
// argint is defined in syscall.c, reads int from tf->esp
int sys_add(void) {
    int a;
    if (argint(0, &a) < 0) return -1;

    int b;
    if (argint(1, &b) < 0) return -1;

    return a + b;
}

// File: add.c
// change atoi to convert negative numbers to int too.
int main(int argc, char **argv) {
    if (argc != 3) {
        printf(2, ``usage: add a b\n''); //error
    } else {
        printf(1, ``%d\n'', add(atoi(argv[1]), atoi(argv[2]));
    }
    exit();
}
\end{lstlisting}

\subsection{Additional details}
\begin{itemize}
    \item A lock is necessary on \texttt{syscalls\_count[]}. This is initialized
        in \texttt{proc.c} and called from \texttt{main.c}.
    \item $syscalls$ like \texttt{ps} and \texttt{send} recquire access to ptable
        and so are defined in \texttt{proc.c} and declared in \texttt{defs.h}
    \item Without \texttt{$\$syscall$.c}, you will not be able to invoke $syscall$ from
        shell.
    \item \texttt{atoi} in \texttt{ulib.c} is changed to handle negative numbers.
\end{itemize}

\section{Inter-process Communication}
We use the $syscall$s \texttt{send} and \texttt{recv} for communication between
the processes. Each process has a personal $message\_queue$; functions like a
mailbox.

\subsection{Unicast}
\begin{itemize}
    \item \texttt{send}:
        \begin{enumerate}
            \item \texttt{acquire}s the lock on \texttt{ptable} as it's a $write$.
            \item $enqueues$ the \texttt{msg} in the \texttt{rec\_pid}'s queue.
            \item On successful $enqueue$, wakes up the process \texttt{rec\_pid}
            \item \texttt{release}s the acquired lock.
        \end{enumerate}
    \item \texttt{recv}:
        \begin{enumerate}
            \item This is a blocking call.
            \item \texttt{acquire}s the lock on \texttt{ptable}.
            \item Tries to $dequeue$ from it's message queue.
            \item If unsuccessful (empty message queue), it puts itself to sleep
                until there's a new message.
            \item On waking up, it again $dequeue$s and returns with the message.
            \item \texttt{release}s the lock on \texttt{ptable}.
        \end{enumerate}
\end{itemize}


%\subsection{Multicast}
%We use the $syscall$s \texttt{send\_multi} and \texttt{recv} for communication
%between the processes.
%\begin{itemize}
%    \item \texttt{send\_multi} $enqueues$ to the $message\_queue$'s.
%    \item \texttt{send\_multi} also sends a signal to these \texttt{rec\_pid}
%    \item \texttt{Preparing signal system:}
%        \begin{enumerate}
%            \item define \texttt{struct signal\_stack} and include one in
%                \texttt{struct proc}
%            \item define \texttt{signal\_handler} and include one in
%                \texttt{struct proc}
%        \end{enumerate}
%\end{itemize}

\newpage
\section{Distributed Algorithm:}
\subsection {$Sum$ using UNICAST}
\begin{itemize}
    \item The $array$ is divided into \texttt{N\_CHILD\_PROCESSES} equal
        contiguous segments.
    \item Each child process computes $partial\_sum$ on it's assigned segment.
    \item Child then sends the result to it's parent using $syscall$ \texttt{send}
    \item Parent collects all the $msg$s from it's child processes
        using $syscall$ \texttt{recv}.
    \item Parent, then, consolidates these $partial\_sum$s into $total\_sum$
\end{itemize}
\begin{lstlisting}

// File: assig1_8.c
// N_PROCESSES == 1 is handled seperately;
int get_sum(short* a, int size) {
    int parent_id = getpid();
    int length_of_segment = size/(N_PROCESSES - 1);

    for (int i = 0; i < N_PROCESSES - 1; ++i) {
        if (!fork()) {
            int start = i * length_of_segment;
            int end = start + length_of_segment;
            if (i == N_PROCESSES - 2) end = size;

            int sum = accumulate_range(a, start, end);
            send(getpid(), parent, itoa(sum));
            exit();
        }
    }

    int ret = consolidated_sum();
    return ret;
}

int consolidated_sum() {
    int ret = 0;
    char *sum = (char *) malloc(MSGSIZE);
    for (int i = 0; i < N_PROCESSES - 1; ++i) {
        recv(sum);
        ret += atoi(sum);
    }
    return ret;
}
\end{lstlisting}
%\subsection {$Variance$ using MULTICAST}
\end{document}
